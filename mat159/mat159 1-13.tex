\documentclass{article}
\usepackage{amsmath,amsfonts}
\usepackage{graphicx} % Required for inserting images

\title{mat159 1-13}
\date{January 2026}

\begin{document}
	
	\maketitle
	\section{Week 2}
	\subsection{Review}
	Indefinite Integrals
	\begin{itemize}
		\item Existence / Uniqueness
		\item Method of integration
		\begin{itemize}
			\item Linearity
			\item formal invariance $\to $ substitution
			\item derivation $\to$ ISP
		\end{itemize}
	\end{itemize}
	Recall:\\
	\[\begin{split}
		d(fg)&=fdg-gdf\\
		&=fg'dx-gf'dx
	\end{split}\]
	\[\implies \int fg'dx=fg-\int gf'dx\]
	\subsubsection{Example}
	$\int\arctan x\ dx$\\
	$=\underbrace{\arctan x*x}_{fg} -\underbrace{\int x\ d\arctan x}_{g df}$\\
	$=\arctan x*x -\frac{1}{2}\int \frac{1}{1+x^2}d(x^2+1)$\\
	$=\arctan x *x -\frac{1}{2}\ln(1+x^2)+C$
	\subsubsection{Example}
	\[\int \underbrace{\ln x}_f \underbrace{dx}_g\]
	\[=\underbrace{x\ln x}_{fg} -\int \underbrace{xd\ln x}_{gdf} \]
	\[=x\ln x -\int x*\frac{1}{x}dx\]
	\[=x(\ln x-1)+C\]
	\subsection{R[x] Polynomial over $\mathbb R$}
	$f=a_nx^n+a_{n-1}x^{n-1}+\dots +a_1x+a_0, \forall 0\leq j\leq n, a_k\in \mathbb R, a_n\not = 0$\\
	$\deg(f)=n$\\
	Special Case: $f=c,\ c\in\mathbb R$, then $\deg(f)=0\text{ or }-\infty$ depending on your use case\\
	Division Algorithm holds for $\mathbb R[x]$\\
	$g|f$ then $f=g*g+v$, where $\deg(v)<\deg(g)$\\
	$\mathbb R$ is not algebraically closed.\\
	Example: $x^2+1$\\
	$\mathbb C$ is algebraically closed.\\
	\begin{itemize}
		\item Every function in $\mathbb C$ has a root in $\mathbb C$
		\item Every root of a function in $\mathbb C$ is in $\mathbb C$
		\item Every function in $\mathbb C$ can be decomposed into linear components.
	\end{itemize}
	\[\mathbb{R\subseteq C}\]
	If $\omega\in \mathbb{C\backslash R}$ is a root of $f=a_xx^n+a_{n-1}x^{n-1}+\dots +a_1x+a_0$, then so is $\overline{\omega}$.\\
	\[f(\omega)=a_n\omega^n+a_{n-1}\omega^{n-1}+\dots+a_1\omega+a_0=0\]
	Take the conjugate on both sides.\\
	\[\overline{f(\omega)}=\overline{a_n\omega^n+a_{n-1}\omega^{n-1}+\dots+a_1\omega+a_0}=\overline{0}\]
	\[\overline{a_n\omega^n+a_{n-1}\omega^{n-1}+\dots+a_1x+a_0}=\overline{0}\]
	$\overline{a+b}=\overline{a}+\overline{b}$ so we can split this up, furthermore, $a_n,a_{n-1},a_{n-2},\dots$ are all in $\mathbb R$, so $\overline{a}=a$.\\
	\[a_n\overline{\omega^n}+a_{n-1}\overline{\omega^{n-1}}+\dots+a_1\overline{\omega}+a_0=0\]
	Thus, the conjugate of $\omega$ is also a root.
\end{document}
