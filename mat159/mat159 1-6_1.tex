\documentclass{article}
\usepackage{graphicx} % Required for inserting images
\usepackage{amsmath,amsfonts}

\title{mat159 1-6}
\date{January 2026}

\begin{document}

\maketitle

\section{Mat159 Analysis II}
Differentiation $\to $ Integration
Start with trying to find the area under a curve, you can draw trapezoids or squares to find the area. However, this raises two questions.\\
1. Does this sum exist?\\
2. Does this sum converge?\\
Newton and Lebniz later proved that you can find the area using
\[F(b)-F(a)=\int_a^bf(x)dx\]
\[\text{where }F'(x)=f(x),\forall x\in(a,b)\]
Liouville (1830) Proved that For elliptical integrals have no primitive, that there is no anti-derivative in finite terms elementry terms. This was problematic for celestial orbits and for pendulums, both important in engineering. \\
\subsection{Primitive (Anti-derivative)}
\noindent Def: Let $D\subseteq \mathbb{R}$ be open\\
$f:D\to\mathbb R$ We say $F:D\to \mathbb R$ is a primitive of f if \\
$\forall x\in D,f(x)=F'(x)$\\
Does anti-derivative always exists and is this anti-derivative unique?\\
Uniqueness: No! If $F$ is a primitive, then so is $F + C$ for $C\in\mathbb R$\\
Prop:\\
Every primitive of f in a connected domain D differ by a Constant (open interval)\\
Proof: Assume that $F_1,F_2:D\to\mathbb R$ are both primitives of f.
\[\forall x\in D,y\in D, y>x\]
Define $G=F_1-F_2$\\
\[G(y)-G(x)=G'(\epsilon)(y-x)\]
This is the $F_1(y)-F_2(y)-F_1(x)+F_2(x)=\int_x^yf(x)dx-\int_x^yf(x)dx$.\\
Recall: Darboux function: A function that permits the intermediate value theorem.\\
Note: A darboux function does not have to be continuous.\\
Theorem: $f:[a,b]\to\mathbb R$ such that $f\in C([a,b])$ and f is differentiable in $(a,b)$. Then f' is Darboux.\\
Proof: To prove that f' is darboux, we just need to prove that for every value between a and b, f' takes on that value.\\
Assume an arbitrary $y\in(a,b)$ where $f'(a)>f'(b)$ without the loss of generality. Consider $g:[a,b]\to\mathbb R$ where $g(x)=f'(x)-yx$\\
Then $g(x)$ achieves and max or min between a and b, not at a because $f'(a)-ya$
\end{document}
