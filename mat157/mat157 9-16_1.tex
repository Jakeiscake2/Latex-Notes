\documentclass{article}
\usepackage{graphicx} % Required for inserting images
\usepackage{amsfonts}

\title{MAT157H5, LEC0101, Sep 16}

\begin{document}

\maketitle

\noindent Review
\begin{itemize}
    \item Sets
    \begin{itemize}
        \item Cartesian Product
        \item $A \times B = \{(x,y)\mid x\in A, y\in B\}$
    \end{itemize}
\end{itemize}
Relation (Binary)
\begin{enumerate}
    \item $x\in A$
    \item $x \geq y$
    \item $A \subseteq B$
    \item $a=b$ mod 4
    \item $\frac{x}{y} \in Q$
    \item $\triangle ABC \cong \triangle DEF$
\end{enumerate}
\noindent Set (Binary Relation)
\\Let A, B be two sets. We say a subset $R\subset A\times B$ a relation from A to B
\\\\Example:\\
Let A and B be two sets 
\begin{itemize}
    \item $\emptyset \subseteq A \times B$ (trivial relation)
    \begin{itemize}
        \item (No relation is a relation) 
    \end{itemize}
    \item $A\times B \subseteq A\times B$ (Universal relation)
\end{itemize}
Example:\\
Take $R = \{(1,3),(2,2),(7,6)\}\subseteq \mathbb{Z} \times \mathbb{Z}$
\\Types of relations
\begin{itemize}
    \item Function
    \item Classification
    \item Order
\end{itemize}
\textbf{Function}
Def: \\

Let $R\subseteq A\times B$ be a binary relation
We say R is\\ 
Entire: if $\forall x \in A, \exists y \in R, st (x,y)\in R$\\
Functional: if $\forall x\in A, y_1,y_2\in B, (x,y_1) \in R \wedge (x,y_2) \in R \rightarrow y_1=y_2$\\
We say a relation $f\subseteq A\times B$ is a function if it is both entire and functional, and we denote it by $f:A\rightarrow B$
\\\\Remark:
For $(x,y)\in R$, we denote it by $xRy$, x relation y
\\If $xfy$ we also denote it by $y=f(x)$
\\$y=f(x)$ is a statement and a claim.
\\$f: A\rightarrow B$ means subset where you can represent a function
\\\\Example:
\\$A\rightarrow B$
\\If there is an element in A which does not have a B, then its not entire
\\If there is an element in A which corresponds to two elements in B, then its not functional
\\Logically Speaking
\\Entire: Every input x has at least an output y (output exists)
\\Functional; Every input x has at most one output y (output is unique)
\\Taken together, every input x has one output y
\\\\Remark
\\A function's definition involves A and B, called the domain and codomain
\\\\Example:
\\$y=x^2$, not a function, no A and B domains
\\$f: R \rightarrow R$, this alone is a function, $x\rightarrow x^2$
\\$f: R\rightarrow [0,\infty)$, $x\rightarrow x^2$, if there is difference in domain or codomain, not the same function
\section{Image and Preimage}
\noindent Let $f: A\rightarrow B$ be a function
\\$\forall c\subseteq A$ the set $f(c)=\{f(x)\mid x\in C\} \subseteq B$
\\is called the image of C under f
\\\\$\forall D \subseteq B$ the set $f^{-1}(D)=\{x\mid f(x)\in D\} \subseteq A$
\\this is called the preimage of D under f

\noindent Remark:
\\$f^{-1}$ does not always mean the inverse
\\$f^{-1}(y), y\in B$ means the inverse 
\\$f^{-1}(\{y\}), y\in B$ mean preimage 

\noindent Example:
\\$f^{-1}(y)$ means there is an inverse, it is bijective (One to one) which is not claimed by knowing its a function.

\noindent Example:
\\$f:R\rightarrow R$ $x\rightarrow x^2$
\\

\begin{table}
    \centering
    \begin{tabular}{|c|c|c|}\hline
         $f^{-1}(c)$&  c& f(c)\\\hline
         {0}&  {0}& {0}\\\hline
         {-1,1}&  {1}& {1}\\\hline
         [-1,1]&  [0,1]& [0,1]\\\hline
         $\emptyset$&  {-1}& {1}\\\hline
    \end{tabular}
    \label{tab:placeholder}
\end{table}
\noindent Theory
\\Let $f:A\rightarrow B$
\\$\forall C\subseteq A$ $C\subseteq f^{-1}(f(C))$
\\$\forall D\subseteq B$ $f(f^{-1}(D))\subseteq D$
\\injectivity and subjectivity
\noindent Proof:
\\Let $C\subseteq A$ be arbitrary
\\Let $x\in C$, then
\\$x\in C \rightarrow f(x) \in f(C)\rightarrow x\in f^{-1}(f(x))$

\noindent Let $D\subseteq B$ be arbitrary
\\Let $y\in f(f^{-1}(D))$, then
\\$\exists x \in f^{-1}(D)$ st $y=f(x)\in D$
\end{document}
