\documentclass{article}
\usepackage{amsmath}
\usepackage{amsfonts}
\usepackage{graphicx} % Required for inserting images

\title{mat157 10-9}
\date{October 2025}

\begin{document}

\maketitle
\section{Week 7}
\subsection{Borel-Lebesgue Theorem}
\noindent Let I=[a,b] be a closed interval, then any open cover (each $\bigcup_a$ is an open interval) permits a FINITE subcover.
\\Proof: Assume to the contrary that $\{\bigcup_x\}_{a\in k}$ is a cover of I, with no finite subcover.
\\Key Idea: Bisection!
\\Imagine a closed interval with [a,b] and $\frac{a+b}{2}$ as the middle.
\\$[a,\frac{a+b}{2}]\subseteq U_{a\in k_1}U_a$ This is the first half.
\\$[\frac{a+b}{2},b]\subseteq \bigcup_{a\in k_2}U_a$ This is the second half.
\\If both are finite, then the union would be finite, thus at least one of them, either the first half or the second half is infinite.
\\Without the loss of generality, assume that it is the first half which is infinite. 
\\We can then split it in two again. Creating a nested closed interval. 
\\By the Cauchy-Cantor Theorem, $\exists c\in \bigcap_{u\in \mathbb{N}}I$
\\Remark: It can be shown that the $\bigcap_{u\in \mathbb{N}}I$ is {c}
\\In particular, $c\in I\subseteq\bigcup_{a\in k}\bigcup_a$
\\$c\in \bigcup_{a^*}$ for some $a^*\in k$
\\There must be some open cover which contains this c. Let the open cover be $(a^*,b^*)$
\\$c-a^*>0$
\\$b^*-c>0$
\\Let l = $\min(c-a^*,b-c)$
\\c is either on the righter or left side.
\\Pick n>>1, so that $|I_n|=\frac{b-a}{2^n}<l$
\\and $I_n\subseteq \bigcup_{a^*}$
\\This contradicts the statement that the smallest part must permit no finite subcover. \\Thus, proof by contradiction.
\section{Bolzano–Weierstrass theorem}
\noindent Tool: Limit point
\\Def: 
\\Let $A\not = \emptyset $ be a subset of $\mathbb{R}$. We say that $p\in \mathbb{R}$ is a limit point of A, if any open neighborhood of p (is an open interval containing p) contains at least one point $x\in A$ in it and $x\not = p$.
\\Remark: If there is "at least one point", there is an infinite amount of points.
\subsection{Example}
\noindent A=[0,1)
\\$L=\{x\in\mathbb{R}|x\text{ is a limit point of A}\}$
\\=[0,1]
\subsection{Example}
\noindent $A=\mathbb{Q}$
\\$L=\{x\in\mathbb{R}|x\text{ is a limit point of A}\}$
\\$=\mathbb{R}$
\subsection{Example}
\noindent $A=[0,1]\cup\{2\}$
\\$L=\{x\in\mathbb{R}|x\text{ is a limit point of A}\}$
\\$=[0,1]$
\subsection{Bolzano–Weierstrass theorem}
\noindent Let A be any infinite set in $\mathbb{R}$
\\If A is bounded, then $L\not = \emptyset$
\\$L=\{x\in\mathbb{R}|x\text{ is a limit point of A}\}$
\\Proof:
\\Since A is bounded, $\exists a,b\in \mathbb{R}$, $a<b$, such that 
\\$A\subseteq [a,b]$
\\For the sake of contradiction, assume that $L_A=\emptyset$
\\Let $x\in [a,b]$, then $x\not \in L_A$
\\Then we can find an open neighborhood $V_x$ of x such that $V_x\cap A$ has at most finitely many elements. 
\\$[a,b] \subseteq \bigcup_{x\in [a,b]}V_x$
\\By Borel-Lebesgue theorem, $\exists k_1,k_2,\dots,k_N\in \mathbb{N}$
\\$A\subseteq [a,b]\subseteq \bigcup_{1\leq i \leq \mathbb{N}}V_{k_i}$
\\However, this implies that A is finite, and by contradiction the proof is true.
\end{document}
