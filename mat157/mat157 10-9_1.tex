\documentclass{article}
\usepackage{amsmath}
\usepackage{amsfonts}
\usepackage{graphicx} % Required for inserting images

\title{mat157 10-9}
\date{October 2025}

\begin{document}

\maketitle
\section{Week 6}
\noindent Dedekind cut ($\alpha=A|A'$
\\$\mathbb{Q}\rightarrow\mathbb{R}$
\\Field
\\(1)->(16) Order
\\Archimedes Property holds
\subsubsection{Remark}
\noindent Archimedes P holds for R.
\\Actually, $\forall a=A|A'\in \mathbb{R}$
\\Pick $y\in A'\subseteq\mathbb{Q}$, then 
\\exists an $z\in \mathbb{N}$ such that z>y
\\Identify z with $z=B|B'$
\\$a<z$
\subsection{Completness of $\mathbb{R}$}
\noindent Dedekind Completeness
\\LUB
\\Cauchy-Contor Theory
\\Borel-Lebesgue theorem
\\Bolzano–Weierstrass theorem
\\Cauchy Completeness

\section{Dedekind Completeness Theorem}
\noindent Tool: Dedekind Cut
\\Theorem: Let $A,B\subseteq \mathbb{R}$ such that B dominates A. ($\forall x\in A,\forall y\in B, x\leq y$)
\\Then $\exists c\in \mathbb{R}$ such that 
\\$\forall x\in A,\forall y\in B, x\leq c\leq y$
\subsection{Remark}
\noindent If A and B are no contacting each other, there is no issue, pick a c.
\\Proof: $x\in A,y\in B$ we use $x=A_x|A'_x,y=B_y|B_y'$ to represent their Dedekind cut, respectively.
\\Let $c=\bigcup_{x\in A} A_x,C'=\mathbb{Q}/C$
\\Note that $A\not = \emptyset, \exists x_0\in A, \text{ thus } A_{x_0}\not \emptyset$
\\Moreover, $B\not = \emptyset, \text{ hence } \exists y_0\in B, \text{ thus }B'_{y_0}\not = \emptyset$
\\$c\cap B'_{y_0}=\emptyset$
\\(2): This C, C' forms a non-trivial parition of Q
\\Let $z\in C=\bigcup_{x\in A}A_x$ and $z^\sim <z$
\\$z\in C\implies z\in A_{x^*}$ for some $x^*\in A$
\\$\implies z^\sim \in A_{x^*}$
\\$\implies z^\sim \in C$
\\Hence, C is closed downwards.
\\(3): Assume to the contrary that $z^*$ is a maximum of C. 
\\$a^-$ portion????, $z^*\in C$
\\This implies that $z^*\in A_{x_1}$ for some $x_1\in A$
\\That implies that $z^*$ is a maximum of $A_{x_!}$ Contradicting that A has no maximum.
\\Thus C does not permit any maximum.
\\(1)+(2)+(3): c=C|C'$\in \mathbb{R}$
\\$\forall x_2\in A, A_{x_2} \subseteq \bigcup_{x\in A}A_x=C$
\\Thus $x_2\leq c$
\\$\forall y\in B, \forall x\in A, A_x\subseteq B_y$
\\$\implies \forall y\in B, C=\bigcup_{x\in A}A_x\subseteq B_y$
\\$\implies \forall y\in B, c\leq y$
\\Remark: c is not unique, but this does give the smallest possible c

\section{Theorem of Least Upper Bound}
\noindent Tool: sup and inf
\\Def: Let A be a non-empty set, A number $s\in \mathbb{R}$ is an upper bound of A if for $\forall x\in A, x\leq s$. 
\\We call the minimum of all upper bounds of A the sup.
\\The opposite of inf
\\Def: Let A be a non-empty set, A number $s\in \mathbb{R}$ is a lower bound of A if for $\forall x\in A, x\geq s$. 
\\We call the maximum of all lower bounds of A the inf.
\subsection{Example}
\noindent $A=(-\infty,1]$ sup A=1
\\$A=(-\infty,1)$ sup A=1
\\$A=\{1+\frac{1}{n}|n\in \mathbb{N}\}$ sup A=1
\subsection{Theorem of Least Upper Bound}
\noindent Any set $A\subseteq \mathbb{R}$ bounded from above permits a least upper bound.
\\Proof:
\\Let $B=\{u\in \mathbb{R}|\text{u is an upper bound of A}\}=\emptyset$
\\Now B dominates A
\\Now, using the Dedekind Completeness Theorem, $\forall x\in A, \forall y\in B, \exists c\in \mathbb{R},x\leq c\leq y$
\\c is now the least upper bound (Yippee!)
\subsection{Remark}
\noindent If sup $A\in A$, then the sup A = max A
\\For a set bounded from above, a maximum may not exists, but a supreme will exist. 
\\Remark:
\\In practice, to prove that sup A = $ x^*$
\\We show two things
\\\begin{enumerate}
    \item $\forall x\in A, x\leq x^*$ ($x^*$ is an upper bound)
    \item $\forall \epsilon > 0, \exists x_\epsilon \in A, st\ x_\epsilon < x-\epsilon$
\end{enumerate}
\subsection{Example}
\noindent Show that sup A = 1, where
\\$A=\{1-\frac{1}{n}|n\in \mathbb{N}\}$
\\Proof: 
\\$\forall x\in A, x=1-\frac{1}{n}$ for some $n\in \mathbb{N}$ thus, 
\\$x\leq 1$
\\\\$\forall \epsilon>0$, pick x = $1-\frac{1}{n^*}$ where $n^*>tba$
\\$1-\frac{1}{n^*}>1-\epsilon$
\\$\frac{1}{n^*}<\epsilon$
\\$n^*>\frac{1}\epsilon$
\\$n^* > \frac{1}{\epsilon}$
\\thus, $x=1-\frac{1}{n^*}>1-\frac{1}{\frac{1}{\epsilon}}=1-\epsilon$
\subsection{Example}
\noindent Let A, B be non-empty sets bounded from above.
\\$A+B=\{x+y\ |\ x\in A,y\in B\}$
\\Prove that sup(A+B) = sup(A)+sup(B)
\\Proof:
\\For any $\forall z \in A+B$, z=x+y for some $x\in A, y\in B$
\\$z\leq $ sup(A) + sup(B)
\\Thus, z is an upper bound
\\$\forall \epsilon>0$
\\$\exists x^*\in A, st\ x^*>$ sup(A) - $\frac{\epsilon}{2}$
\\$\exists y^*\in B, st\ y^*>$ sup(B) - $\frac{\epsilon}{2}$
\\It follows that 
\\$x^*+y^*>supA+supB-\epsilon$
\section{Cauchy-Cantor Theorem}
\noindent Tool: Closed Nested Interval
\\Def: We say $\{A_u\}_{a\in \mathbb{N+}}$ a nested set if, 
\\$A_1\supseteq A_2\supseteq A_3\supseteq \dots \supseteq A_n\supseteq\dots$
\\If each set is a closed interval. ($A_n=[a,b]$)
\\then we call it a nested closed interval.
\subsection{Cauchy-Cantor Theorem}
\noindent Let ${I_k}_{k\in \mathbb{N}+}$ be a nested closed interval,
\\Then $\bigcap_{k \in \mathbb{N+}}I_k\not =\emptyset$
\\Proof:
\\$\forall n,m\in \mathbb{N}+, (\text{without the loss of generality})\ n\leq m, I_n\supseteq I_m$
\\$a_n\leq b_m, a_m\leq b_m$
\\Thus, each $b_k$ bounds each $a_l, \forall k,l\in \mathbb{N}+$ ($b_k\geq a_l$)
\\Now, let $A=\{a_k|k\in \mathbb{N}+\}$
\\$B=\{b_k|k\in\mathbb{N}+\}$
\\Then, 
\\$\forall l\in\mathbb{N}+, a_l\leq \sup A\leq \inf B\leq b_l$
\subsection{Remark}
\noindent The actually tells you that [sup A, inf B] = $\bigcap_{k\in \mathbb{N}+}I_k$
\section{Borel-Lebesgue Theorem}
\noindent Tools: Open cover
\\Def: A family of sets $\{\bigcup_a\}_{a\in k}$ is called a cover of a set A, if $A\subseteq \bigcup _{a\in k}\bigcup_a$ 
\\The set of $\{\bigcup_a\}_{a\in V}\ V\subseteq k$ is a sub-cover, if $A\subseteq \bigcup_{a\in \mathbb{N}}\bigcup_a$
\subsection{Example}
\noindent Let A=[0,1]
\\k={1} $\bigcup_1=(-1,0)\ A\subseteq \bigcup_1$
\\k={1,2} $\bigcup_1=(-1,1)\ \bigcup_2=(0,2), A\subseteq \bigcup_{a\in \{1,2\}}\bigcup_a$
\\$\forall k\in\mathbb{N}+,\bigcup _k=(-\frac{1}{k},1], A\subseteq \bigcup_{k\in \mathbb{N}+}\bigcup_k$
\\$A\subseteq \bigcup_{x\in [0,1]}\{x\}$
\subsection{Borel-Lebesgue Theorem}
\noindent Let I=[a,b] be a closed interval, then any open cover (each $\bigcup_a$ is an open interval) permits a FINITE subcover.
\\Proof: Assume to the contrary that $\{\bigcup_x\}_{a\in k}$ is a cover of I, with no finite subcover.
\\Key Idea: Bisection!
\end{document}
