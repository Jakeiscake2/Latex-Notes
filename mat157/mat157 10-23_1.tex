\documentclass{article}
\usepackage{amsmath}
\usepackage{amsfonts}
\usepackage{graphicx} % Required for inserting images

\title{mat157 10-23}
\date{October 2025}

\begin{document}

\maketitle

\noindent Week 8:
\section{Review}
\begin{itemize}
    \item For the real numbers, we have discussed order, arithemtic, and completeness
\end{itemize}
\section{Notes}
\noindent Proposition:
\\Let $\{a_n\}_{n\in\mathbb{N}},\{b_n\}_{n\in\mathbb{N}}$ be sequences such that 
\[\lim_{n\to\infty}a_n=a,\lim_{n\to\infty}b_n=b\]
\\$1^0$ If b$>$a, then $b_n$ dominates $a_n$ Strictly eventually if $\exists N_0\in\mathbb{N}:\forall n>N_0, b_n>a_n$
\\$2^0$ If $b_n\text{ dominates }a_n$, then $b\geq a$
\\Proof:
\\$1^0$ Let $\epsilon=\frac{b-a}{2}>0$
\\Then $\exists N_1,N_2\in\mathbb{N}$
\\$\forall n>N,|a_n-a|<\epsilon$
\\$\forall n>N_2,|b_n-b|<\epsilon$
\\$N_3=\max(N_1,N_2)$
\\$\implies\forall n>N_3,b_n>b-\epsilon=a+\epsilon>a_n$
\\$2^0$ Assume to the contrary $\frac{b-a}{2}$ is not greater than 0, then $b<a$, contradicting the result from 1.
\section{Remark}
\noindent For the statement $1^0$, $b>a$ cannot be replaced by $b\geq a$
\\Example:
\[a_n=1-\frac{1}{n^2},b_n=1-\frac{1}{n}\]
\\In this case there is no dominance, both the limits are equal to one.
\\Example:
\[a_n=(-1)^n(\frac{1}{n}),b_n=0\]
\\The limit of both the sequences are equal to 0, but there is no dominance.
\subsection{Remark}
\noindent A special common scenario: asymptotic sign definition: 
\\Let $\lim a_n=a\in\mathbb{R}$
\\If a>0, eventually $a_n>a-\epsilon>0$ for some $\epsilon<<1$
\\If a<0, eventually $a_n<a-\epsilon<0$ for some $\epsilon<<1$
\\If a=0, no sign definitions, for example $a_n=(-1)^n(\frac{1}{n})$
\subsection{Corollary}
\noindent Let $\{a_n\}_{n\in\mathbb{N}},\{b_n\}_{n\in\mathbb{N}},\{c_n\}_{n\in\mathbb{N}}$ be sequences such that
\\$\forall n\in\mathbb{N},a_n\leq b_n\leq c_n$
\[\lim_{n\to\infty}a_n=\lim_{n\to\infty}c_n=L\in\mathbb{R}\implies \lim_{n\to\infty}b_n=L\]
\\You know not only then that $b_n$ converges, but also that it converges to L.
\\This is squeeze theorem
\\We know that if $\{a_n\}_{n\in\mathbb{N}}$ is bounded and $\{b_n\}_{n\in\mathbb{N}}$ is convergent. Then $\{a_nb_n\}_{n\in\mathbb{N}}$ is bounded but may not be convergent.
\subsection{Example}
\noindent $a_n=(-1)^n,b_n=1,a_nb_n=(-1)^n$, this is bounded but divergent.
\\Proposition:
\\if $\{a_n\}_{n\in\mathbb{N}}$ is bounded and $\lim_{n\to\infty}b_n=0$.
\\Then $\lim_{n\to\infty}a_nb_n=0$
\\Proof:
\\Let $M\in\mathbb{R}_{\geq 0}$, be the bound of $\{a_n\}_{n\in\mathbb{N}}$. $\forall n\in\mathbb{N},|a_n|\leq M$
\[0=a_n\leq |a_nb_n|\leq M|b_n|=0\]
\\By squeeze theorem.
\subsection{Example}
\noindent Let $\{a_n\}_{n\in\mathbb{N}}$ be a positive sequence, $k\in\mathbb{N}_+$ be fixed
\[\lim_{n\to\infty}a_n^k=L^k\text{ assuming }L> 0\implies \lim_{n\to\infty}a_n=L\]
\\Proof:
\\Recall: 
\[a^k-b^k=(a-b)(a^{k-1}+a^{k-2}b-a^{k-2}b^{2}+\dots+ab^{k-2}+b^{k-1})\]
\[(a^k-L^k)=(a_n-L)(\Sigma_{i=1}^{k}a_n^{k-i}L^{i-1})\]
\\Case 1: If L=0, then the result is evident, converge $\epsilon$ to 0 to show this.
\\Case 2: $L>0$, Then we know that $\exists N_0\in \mathbb{N}$ such that $\forall n>N_0, a_n^k>(\frac{L}{2})^k\text{, so }a_n>\frac{L}{2}>0$
\\Knowing this, $a_n-L$ converges to 0, thus $a_n=L$
\subsection{Remark}
\noindent We know 
\[\lim_{n\to\infty}a_n=0,|b_n|\leq M\implies \lim_{n\to\infty}a_nb_n=0\]
\\On the other hand
\\$c_n=a_nb_n$ such that $\lim_{n\to\infty}c=0$, $|b_n|\geq b_n>0$
\\Then 
\[a_n=c_n\frac{1}{b_n}\]
\subsection{Example}
\noindent Find $\lim_{n\to\infty}\sin(\frac{1}{n})$
\\Geometrically, $0<\sin(\frac{1}{n})<\frac{1}{n}$
\\Thus, $0\leq \lim_{n\to\infty}\sin(\frac{1}{n})\leq \lim_{n\to\infty}\frac{1}{n} \leq 0$
\[\lim_{n\to\infty}\sin(\frac{1}{n})=0\]
\subsection{Example}
\noindent Find $\lim_{n\to\infty}\cos(\frac{1}{n})$
\\$\lim_{n\to\infty}\sin(\frac{1}{n})=0\implies \lim_{n\to\infty}\sin^2(\frac{1}{n})=0\implies \lim_{n\to\infty}1-\sin^2(\frac{1}{n})=1$
\\$\implies \lim_{n\to\infty}\cos(\frac{1}{n})=\sqrt{1}=1$
\subsection{Infinity of the limit}
\noindent Definition: Let $\{a_n\}_{n\in\mathbb{N}}$
\\We say that $\{a_n\}_{n\in\mathbb{N}}$ tends to infinity if 
\[\forall M>0,\exists N\in\mathbb{N}: \forall n>N, a_n>M\]
\\We denote it by $\lim_{n\to\infty}a_n=+\infty$
\subsection{Example:}
\[a_n=n,\lim_{n\to\infty}a_n=\infty\]
\[a_n=-n,\lim_{n\to\infty}a_n=-\infty\]
\[a_n=(-1)^nn,\lim_{n\to\infty}a_n=\text{DNE}\]
\subsection{Indefinite Type}
\noindent Typical Examples Include:
\[\frac{\infty}{\infty},\infty-\infty,\frac{0}{0},0*\infty,1^\infty\]
\subsection{Example}:
\[\lim_{n\to\infty}\frac{a_pn^p+a_{q-1}n^{p-1}+\dots+a_1n+a_0}{b_qn^q+b_{q-1}n^{q-1}+\dots+b_1n+b_0}\]
\\Where $a_p\not=0,b_p\not=0$
\\If $p<q$, the result is 0
\\If $p>q$, the result is $\pm \infty$
\\If $p=q$, the result is $\frac{a_p}{b_p}$
\subsection{Example}
\noindent $\lim_{n\to\infty}\frac{n^p}{a^n}$ where $a>1,p\in\mathbb{N}$
\\Since $a>1,$ let $a=1-\lambda,\lambda>0$
\\Case 1: p=1
\[\frac{n}{a^n}=\frac{n}{(1-\lambda)^n}=\frac{n}{\Sigma_{k=0}^{n}C_n^k\lambda^k}=\frac{n}{1+n\lambda+\frac{n(k-1)}{2}\lambda^2}\]
\[\implies \lim_{n\to\infty}\frac{n}{a^n}=0\]
\\Case 2: p>1
\[\frac{n^p}{a^n}=(\frac{n}{a^{\frac{n}{p}}})^p=(\frac{n}{b})^p, \text{ where }b=a^\frac{1}{p}>1\]
\[\lim_{n\to\infty}\frac{n^p}{a^n}=(\lim_{n\to\infty}\frac{n}{b^n})(\lim_{n\to\infty}\frac{n}{b^n})\dots(\lim_{n\to\infty}\frac{n}{b^n})=0\]
\subsection{Example}
\[\lim_{n\to\infty}\frac{a^n}{n!},a>1\]
\\We know that $\exists N\in\mathbb{N}:\forall k<N,\frac{a}{k}\leq\frac{1}{2}$
\[\frac{a^n}{n!}=\frac{a}{1}\frac{a}{2}\frac{a}{3}\dots\frac{a}{N}\frac{a}{N+1}\dots\frac{a}{n}=S\]
\\Let M = $\frac{a}{1}\frac{a}{2}\frac{a}{3}\dots\frac{a}{N}$
\[0\leq S< \lim_{n\to\infty}M(\frac{1}{2})^{n-N}=0\]
\[\implies \lim_{n\to\infty}\frac{a^n}{n!}=0\]
\subsection{Example}
\[\lim_{n\to\infty}\frac{n!}{n^n}\]
\[\frac{1*2*3*4*\dots*|\frac{n}{2}|}{n*n*n*n*\dots*n}<\frac{1}{n}*\frac{2}{n}\dots*\frac{|\frac{n}{2}|}{2}\leq(\frac{1}{2})^\frac{n}{2}\]
\[<(\frac{1}{2})^{\frac{n}{2}-1}\to 0\]
\\(tbh i have no clue what the prof did here, just search up a proof maybe)
\subsection{Example}
\[\lim_{n\to\infty}n\sin(\frac{1}{n})\]
\[n\sin(\frac{1}{n})\leq \frac{\sin(\frac{1}{n})}{\frac{1}{n}}\]
\[\cos(\frac{1}{n})=\frac{\sin(\frac{1}{n})}{\tan(\frac{1}{n})}\leq n\sin(\frac{1}{n})\leq \frac{\sin(\frac{1}{n})}{\frac{1}{n}}\leq \frac{\frac{1}{n}}{\frac{1}{n}=1}\]
\\Thus, $n\sin(\frac{1}{n})$
\subsection{Example}
\[\lim_{n\to\infty}(1+\frac{1}{n})^n\]
\\Claim: (TO DO: HOMEWORK) A monotonically increasing sequences that is bounded from above converges.


\end{document}
