\documentclass{article}
\usepackage{graphicx} % Required for inserting images
\usepackage{amsfonts,amsmath}

\title{mat157 11-13}
\date{November 2025}

\begin{document}

\maketitle
\subsection{Review}
\noindent Continuous Function
\\\begin{itemize}
    \item Local: $\lim_{x\to x_0}f(x)=f(x_0)$\\
    The behavior near $x_0$ is the same as behavior at $x_0$\\
    We then have Arithmetic, Order, \\
    $\lim_{n\to \infty}a_n=x_0\implies \lim_{n\to\infty}f(a_n)=f(\lim_{n\to\infty}a_n)$\\
    ,Local boundedness and local sign definiteness.\\
    \item Global: f is continuous everywhere
\end{itemize}
\noindent Infinity as the limit
$f:\mathring{I}_c(x_0)\to\mathbb R$
\[\lim_{x\to x_0}f(x_0)=+\infty\]
\[\iff \forall M>0,\exists \delta >0, st\ \forall x\in \mathring{I}_\delta(x_0),f(x)>M\]
Proposition:
\\Let $f:(a,b)\to \mathbb R$ such that
\begin{enumerate}
    \item f is bounded from above on $(a,b)$
    \item f is monotonically increasing
\end{enumerate}
Then $\lim_{x\to b^-}f(x)$ exists.\\
Note if f is bounded above on $\mathring{I}_c(x_0)$\\
Let $g:(a,c)\to\mathbb R$, then g is well defined for the punctured interval.\\
$\delta \to \sup f(x),x\in\mathring{I}_c(x_0)$\\
If $\delta$ decreases, $g(\delta)$ decreases.\\
If f is also bounded from below, then $\lim_{\delta\to0+}g(\delta)$ exists.\\
This limit is called the limit superior of f and $x_0$ and is denoted by 
\[\lim_{x\to x_0}\sup f\]
Theorem:\\
\[\lim_{x\to x_0}f(x)\text{ exists }\iff \lim_{x\to~x_0-}f(x)=\lim_{x\to~x_0+}f(x)\iff \lim_{x\to x_0}\sup f=\lim_{x\to x_0}\inf f\]
\subsection{Uniform Continuity}
\noindent Def:
\\Let $f:A\to\mathbb R$ be a real valued function, with $A\subseteq \mathbb R$. We know f is uniformly continuous on A if 
\[\forall \epsilon>0,\exists \delta>0, \text{st } \forall x_0\in A, \forall x\in I_\delta(x_0),|f(x)-f(x_0)|<\epsilon\]
The difference here from continuity is the definition of $x_0$ after $\delta$
\[\forall x_0\in A,\forall \epsilon>0,\exists \delta>0, \text{st } \forall x\in I_\delta(x_0),|f(x)-f(x_0)|<\epsilon\]
\subsection{Example}
\[f:(0,+\infty)\to\mathbb R,\ f(x)=\frac{1}{x}\]
Looking at $\delta$, we see that as $x_0$ approaches 0, we need to pick a smaller $\delta$. \\As our $\delta$ decreases, our $\delta$ converges to 0.  
\[g:(0,+\infty)\to\mathbb R,\ f(x)=x^2\]
As $x_0$ increases, our $\delta$ also decreases because the values become larger and larger. 
\subsection{Topology in $\mathbb R$}
\noindent Def:\\
Let $A\subseteq \mathbb R$, we say that $x_0\in A$ is a/an
\begin{itemize}
    \item interior point: if $\exists \delta >0,st\ (x_0-\delta,x_0+\delta)\subseteq A$
    \item boundary point: if $\forall \delta>0,(x_0-\delta,x_0+\delta)\cap A\not = \emptyset,(x_0-\delta,x_0+\delta)\cap A^c\not = \emptyset$
\end{itemize}
\noindent $\mathring S=\{x\in \mathbb R,\text{ x is an interior point of S}\}$
\\$\partial S=\{x\in\mathbb R,\text{ x is a boundary point of S}\}$
\\ Def: A set $S\subseteq \mathbb R$ is called
\begin{itemize}
    \item open, if $S\cap \partial S=\emptyset, ie\ (S=\mathring S)$
    \item Closed if $\partial S\subseteq S$
\end{itemize}
\noindent Proposition: \\
S is open $\iff \mathring S$ is closed\\
Proof:\\
\[S\text{ is open}\implies S\cap \partial S=\emptyset\implies \partial S\subseteq S^c \implies \partial S^c\subseteq S^c\]
Proposition:\\
The union of finitely many open sets is open.\\
The union of finitely many closed sets is closed.\\
Proof:\\
Let $\{O_i\}_{1\leq i \leq n}$ be a finite collection of open sets.\\
\[\forall x_0\in \bigcup_{i=1}^n O_i\implies x_0\in O_{i^*},1\leq i^*\leq n\text{ for some }i^*\]
\[\text{and }\exists \delta >0, st\ I_\delta(x_0)\subseteq O_{i^*}, \text{ thus } I_\delta(x_0)\subseteq \bigcup_{i=1}^nO_i\]
Now let $C_i,1\leq i \leq n$ for a finite collection of closed sets, and consider $(\bigcup_{i=1}^nC_i)^c=\bigcap_{i=1}^nC_i^c$.\\
If $x_0\in \bigcap_{i=1}^nC^c_i\implies \forall 1\leq i \leq n. x\in C_i^c$ and $\exists \delta_i>0$ such that $I_{\delta_i}(x_0)\subseteq C_i$. \\
Pick $\delta = \lim_{1\leq i \leq n}\delta_i>0$ and $I_\delta(x_0)\subseteq C_i$, thus $C_i$ is open and $C_i^c$ is closed.
\subsection{Example}
\noindent Let $A_n=[\frac{1}{n},1-\frac{1}{n}],\ n\geq 1$\\
\[\bigcup_{i=1}^n A_i=(0,1)\]
Let $A_i=(-\frac{1}n,\frac{1}{n})$
\[\bigcap_{i=1}^n A_i={0}\]
$\emptyset$ is vacuously closed and open.\\
$\mathbb R$ is open and closed.\\
Def:\\
Let A be a set. We say that A is disconnected if $\exists$ open sets $U,V\not =\emptyset$ if\\
$U\cap V=\emptyset$\\
$U\cap A\not = \emptyset, V\cap A \not = \emptyset$\\
$A\subseteq U\cup V$\\
If A is not disconnected, it is connected
\subsection{Example}
\noindent$ A = (0,1)\cup (2,3)$ is open\\
$B= [0,1)\cup [2,3]$ is open, take $(-1,1)\cup (1.9,3.1)$ for example\\
Proposition:\\
Every connected set in $\mathbb R$ is an interval.\\
Every open set in $\mathbb R$ is a countable union of disjoint open intervals
\[A=\bigcup_{i\in \delta}(a_i,b_i)\]
A connected component of an open set is one of the intervals in this union.\\
Infinite interval: At least one connected component stretches to infinity. \\
Proof:\\
Start with $A\subseteq \bigcup_{i\in \delta}(a_i,b_i)$ \\
\[\forall x\in A, \exists \delta >0, st (x-\delta,x+\delta)\subseteq A\]
Take $S_s$ and $S_l$ such that 
\[S_s=\{y\in \mathbb R|y>x,(x,y)\subseteq A\}\]
If $S_s$ is unbounded, ie the open interval containing x is unbounded from above, take $b=+\infty$.\\
Otherwise $S_s$ is bounded, then take $b=\sup(S)$\\
\[S_l=\{z\in \mathbb R|z<x,(z,x)\subseteq A\}\]
If $S_l$ is unbounded, ie the open interval containing x is unbounded from below, take $a=-\infty$.\\
Otherwise $S_l$ is bounded, then take $a=\inf(S)$\\

We can now assemble a union of open sets which contains all elements of A, with many repeats and we can see that 
\[A\subseteq \bigcup_{i\in \delta}(a_i,b_i)\]
Before continuing, we can prove that $\bigcup_{i\in \delta}(a_i,b_i)$ is countable. Take the largest rational number in the set, which there always is because $a_i<b_i$. This creates an injective function which maps each open interval to $\mathbb Q$ 

\[\exists U,V,st\ U\cap V=\emptyset\wedge U\cap A\not = \emptyset\wedge V\cap A\not = \emptyset\wedge A\subseteq U\cup V\]
\[\forall U,V\text{ open sets}, \]
\end{document}
