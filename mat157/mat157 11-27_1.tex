\documentclass{article}
\usepackage{graphicx} % Required for inserting images
\usepackage{amsfonts,amsmath}
\title{mat157 11-27}
\date{November 2025}

\begin{document}

\maketitle
\subsection{Review}
\noindent Derivative:\\
Arithmetic\\
Chain Rule\\
$\lim_{\Delta x\to 0}\frac{f(x+\Delta x)-f(x)}{\Delta x}=f'(x)$\\
Differentiation\\
$f(x+\Delta x)=f(x)+A\Delta x+o(\Delta x)$\\
\subsection{Invariance of differentiation}
\noindent $f=f(u)$ where $u=g(u)$, then $f=g(f(v))=f(v)$. However, we also know that $df=f'(u)du$, thus $f'(u)g'(v)fv$\\
\subsection{Hopital Principle?}
\noindent Apparently he bought it?\\
Take the fraction of the limit of two functions. If both give values you have no issue, however if both the top and bottom are 0. You have an issue.\\
\[\lim_{x\to x_0}\frac{f(x)}{g(x)}=\lim_{x\to x_0}\frac 0 0\]
Which is to say $\lim_{x\to x_0}f(x)=0$ and $\lim_{x\to x_0}g(x) = 0$\\
If the derivative of f and g both exist, when can take
\[\lim_{\Delta x\to 0}\frac{f(x+\Delta x)}{g(x+\Delta x)}=\lim_{\Delta x\to 0}\frac{f(x_0)+f'(x_0)\Delta x+o(\Delta x)}{g(x_0)+g'(x_0)\Delta x+o(\Delta x)}\]
\[=\lim_{\Delta x\to 0}\frac{f'(x_0)+o(\Delta x)/\Delta x}{g'(x_0)+o(\Delta x)/\Delta x}\]
\subsection{Properties of Differentiable functions}
\begin{itemize}
    \item Fermat's Theorem
    \item Rolle Theorem
    \item Lagrange Theorem
    \item Cauchy Theorem
\end{itemize}
\noindent The Lagrange Theorem is most useful.\\
\subsection{Fermat Theorem}
\noindent Let $f:D\subseteq \mathbb{R\to R}$ be a function, and f is differentiable at some $x_0\in \mathring{D}$\\
If $f'(x_0)>0$, then f is an increasing function locally.\\
If $f'(x_0)<0$, then f is an decreasing function locally.\\
Proof:\\
Since f is differentiable at $x_0$, 
\[f(x+\Delta x)=f(x)+f'(x)\Delta x+o(\Delta x)\]
Consider that $\exists \delta >0$ such that $\forall |\Delta x|<\delta$.
\[|f(x+\Delta x)-f(x)-f'(x)\Delta x|=|o(\Delta x)|<\frac{1}{2}|f'(x)\Delta x|\]
If $\Delta x > 0$, then 
\[f(x+\Delta x)-f(x)=f'(x_0)\Delta x+o(\Delta x)>\frac{1}{2}|f'(x)\Delta x|>0\]
Since the value of $f(x+\Delta x)-f(x)>0$, it is locally increasing. The same proof can be used for decreasing.\\
\subsubsection{Remark}
\noindent If $f'(x_0)=0$, then it is inconclusive.
\\Example:\\
$f:\mathbb{R\to R},\ x\mapsto x^3$\\
$f'(x_0)=\lim_{x\to x_0}3x^2=0$, but it is strictly increasing at 0
$f:\mathbb{R\to R},\ x\mapsto -x^3$\\
$f'(x_0)=\lim_{x\to x_0}-3x^2=0$, but it is strictly decreasing.\\
Even if $f'(x_0)$ does not exist, it may still be monotonic. For example
$f:\mathbb{R^+\to R},\ x\mapsto x$\\
\subsection{Rolle Theorem}
\noindent Let $f:[a,b]\to \mathbb R$ be a function satisfying 
\begin{enumerate}
    \item f is continuous on $[a,b]$
    \item f is differentiable on $(a,b)$
    \item $f(a)=f(b)$
\end{enumerate}
\noindent Then we know that $\exists c\in (a,b)$ such that $f'(c)=0$
\subsection{Properties of Differentiable function}
\noindent Proof: Since f is continuous on $[a,b]$, so it allows a finite maximum and minimum. \\
Let $M=f(c_1),m=f(c_2)$ and $c_1,c_2\in[a,b]$\\
If $M=m$, then f is a constant function ($f=c$) on $[a,b]$. Take any $c\in[a,b]$ and $f'(c)=0$\\
If $M>m$, at least one of $M$ or $m$ is an interior point because $f(a)=f(b)$.\\
\subsubsection{Lemma:(Corollary of Fermat Theorem)}
\noindent Let $f:D\subseteq \mathbb{R\to R},\ x_0\in\mathring{D}$, Moreover, $x_0$ is locally maximum and $f'(x_0)$ exists. Then $f'(x_0)=0$.\\
Proof:\\
If $f'(x_0)>0$, then $f(x_0+\Delta x)>f(x_0)$.\\
If $f'(x_0)<0$, then $f(x_0-\Delta x)>f(x_0)$.\\
Thus, $f'(x_0)=0$
\subsection{Remark}
\noindent If you do not have the condition $f(a)=f(b)$, it could be a linear function where the max is $a$ and the min is $b$. Thus, if both points are boundary points it does not hold.\\
Differentiability is important because if $f(x)=|x|$, the point $c$ could have $f'(c)$ does not exist.\\
You do not need the boundary points $a$ and $b$ to be differentiable, though. \\Consider $\sqrt{1-x^{2}}$\\
\subsection{Lagrange Theorem}
\noindent $f:[a,b]\to \mathbb R$\\
f is continuous on $[a,b]$\\
f is differentiable on $(a,b)$\\
Then $\exists c\in(a,b)$ such that $f(b)-f(a)=f'(c)(b-a)$\\
The idea is that you find some tangent of the curve equals the slope from $a$ to $b$. $\frac{f(b)-f(a)}{b-a}$.\\
Proof:\\
Define $g:[a,b]\to \mathbb R$\\
$g(c)=(x-a)\frac{f(b)-f(a)}{b-a}-f(x)$\\
g is continuous on $[a,b]$ because it is composed of elementary functions\\
g is differentiable on (a,b) because uhhh idk\\
g(b)=g(a)=-f(a) by substitution. \\
Thus, $\exists c\in(a,b)$ such that $g(c)=0$\\
Then $0=(x-a)\frac{f(b)-f(a)}{b-a}-f(x)\implies f(c)=\frac{f(b)-f(a)}{b-a}$.\\
\subsubsection{Remark}
\noindent Lagrange finite increment formula.
\[f(x+\Delta x)-f(x)=f'(c)\Delta x=f'(x+\theta\Delta x)\]
\subsubsection{Example}
\noindent Recall we have discussed the continuity of $f:(0,+\infty)\to(0,+\infty),\ x\mapsto \frac{1}{x}$\\
Recall that f is point wise continuous but not uniformly continuous.\\
By the Lagrange theorem.
\[f(b)-f(a)=f'(c)(b-a)\]
We can compute the value derivative of f at c\\
\[=-\frac{1}{c^2}(b-a)=-\frac{1}{(a-\theta(b-a))^2}(b-a)\implies\]
\[|f(b)-f(a)|=|\frac{1}{a-\theta(b-a)^2}||y-x|\]
\subsection{Corollary Theorem}
\noindent If $f\in C'([a,b])$, then f is lipschitz.\\
\subsection{Cauchy Theorem}
\noindent $f,g\in[a,b]\to\mathbb R$\\
f,g are continuous
f,g are differentiable
Then, $\exists c\in[a,b]$ such that $(f(b)-f(a))g'(c)=(g(b)-g(a))f'(c)$.\\

\subsection{Higher Order Derivatives and Taylor Expansion}
\noindent $f:\mathbb{R\to R}$ where f is smooth (infinitely differentiable) 
\[f(x)=\sum_{n=0}^\infty a_n x^n=a_0+a_1x+a_2x^2+a_3x^3+\dots\]
\[f(0)=a\]
\[f'(x)=a_1+a_2x+a_3x^2+\dots\]
\[f'(0)=a_2\]
\[f''(x)=2a_2+3*2a_3x+4*3*a_4x^2+\dots\]
\[f''(0)=2a_2\implies a_2=\frac{f''(0)}{2}\]
\[a_n=\frac{f^n(0)}{n!}\]
\subsection{Theorem}
\noindent $f:D\subseteq \mathbb R\to \mathbb R$\\
If f has a derivative up to order n near $x_0\in D$\\
Then,
\[f(x)=\sum_\]
\end{document}
