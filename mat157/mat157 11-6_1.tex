\documentclass{article}
\usepackage{graphicx} % Required for inserting images
\usepackage{amsfonts}
\usepackage{amsmath}

\title{mat157 11-6}
\date{November 6, 2025}

\begin{document}

\maketitle

\section{Week 9}
\noindent Review
\begin{itemize}
    \item Sequences $\{a_n\}_{n\in\mathbb{N}}$\\
    A sequence can be seen as a function from $\mathbb{N\to R}$
    \item limit of a function at a point $x_0$
\end{itemize}
\noindent \textbf{Def:} 
\\We say an interval of the form 
\[\mathring{I}_c(x_0):=(x_0-c,x_0)\cup (x_0,x_0+c), \forall c>0\]
\\This is a punctured interval centered at x.
\\We can use this to analyze a function at a point which does not include that point
\\\textbf{Def}:
\\Let $f:A\subseteq \mathbb{R\to R}:\mathring{I}_c(x_0)\subseteq A$ for some $x_0,c$
\\We say that the limit of f at $x_0$ is $L\in\mathbb{R}$ if 
\[\forall \epsilon>0,\exists \delta >0: \forall x\in \mathring{I}_\delta(x_0),|f(x)-L|<\epsilon\]
\\$0<|x-x_0|<\delta$
\\\textbf{Remark}
\\$\epsilon$ an error bound fixed picked first
\\$\delta $ the level of tolerance permitted afterwards
\\Thus, we will often pick $\delta$ based on $\epsilon$
\\\textbf{Remark}
\\To denote the limit of  f at $x_0$
\\\begin{itemize}
    \item We need f to be defined "near" $x_0$ (Some punctured interval of $x_0$)
    \item We do NOT need f to be defined at $x_0$
\end{itemize}
\noindent \textbf{Remark}
\\We denote the limit by 
\[\lim_{x\to x_0}f(x)=L\]
\\\textbf{Example}
\\$f:\mathbb{R\to R},\ f(x)=c$
\[\forall \epsilon>0, \text{pick }\delta = 1, \forall x\in\mathbb{R}\]
\[0<|x-x_0|<1\implies |f(x)-c=0|<\epsilon\]
\\Thus, 
\[\lim_{x\to x_0}f(x)=c\]
\[f:\mathbb{R\to R}, f(x)=x\]
\[\forall \epsilon>0,\text{pick }\delta = \frac{\epsilon}{2},\forall x\in\mathbb{R}\]
\[0<|x-x_0|<\frac{\epsilon}{2}\implies|f(x)-x_0|=|x-x_0|<\frac{\epsilon}{2}<\epsilon\]
\\Thus,
\[\lim_{x\to x_0}f(x)=x\]
\[f:\mathbb{R\to R}:f(x)=x^2\]
\[\forall \epsilon>0,\text{pick }\delta = TBA, \forall x\in\mathbb{R}\]
\[0<|x-x_0|<\delta\implies |f(x)-x_0^2\]
\[f(x)-x^2=x^2-x_0^2=(x-x_0)(x+x_0)\]
\[|(x-x_0)(x+x_0)|<\delta(|x|+|x_0|)<\delta(|2x_0|+\delta)<\epsilon\]
\\To pick $\delta$, try to set $\delta|2x_0|<\frac{\epsilon}{2}$ and $\delta^2<\frac{\epsilon}{2}$
\[\delta_1=\sqrt{\frac{\epsilon}{2}}\]
\[\delta_2=\frac{\epsilon}{2|x_0|}\]
\\If $x_0$ is 0, $\delta_2$ is not well defined. Make $\delta_2$ smaller
\[\delta_2=\frac{\epsilon}{2|x_0|+1}\]
\[\delta = \delta_1+\delta_2\]
\\\textbf{Example}
\\Heaviside function
\[f:\mathbb{R\to R}\]
\\$f(x)=\begin{cases}
    0,\ x<0\\
    1,\ x\geq 0
\end{cases}$
\noindent Claim: f does not have a limit at 0
\\This means, 
\[\forall L\in\mathbb{R},\exists \epsilon>0,\forall \delta >0,\exists x\in\mathring{I}_\delta(x_0)\ st \ |f(x)-L|\geq \epsilon\]
\\Proof: Given any $L\in\mathbb{R}$, pick $\epsilon=\frac{1}{3},\forall \delta >0,$ pick 
\[x=\begin{cases}
    -\frac{\delta}{2}, \text{if } L\geq \frac{1}{2}\\
    \frac{\delta}{2}, \text{if }L<\frac{1}{2}
\end{cases}\implies 0<|f(x)-0|<L\]
\\Dirichlet function
\[f:\mathbb{R\to R}\]
\[f(x)=\begin{cases}
    1, x\in\mathbb{Q}\\
    0,x\in\mathbb{Q_c}
\end{cases}\]
\\In any interval, there is always going to be a rational and irrational number. Thus, this function does not have a limit anywhere
\\Thomae-Riemann function
\[f:[0,1]\to [0,1]\]
\[f(x)=\begin{cases}
    \frac{1}{q},\ x=\frac{p}{q}, p\in\mathbb{N},q\in\mathbb{N}_+,\gcd(p,q)=1\\
    0,\ x\not \in \mathbb{Q}
\end{cases}\]
\\Note: limit at any point is 0
\\There is finitely many possible numbers for each $\frac{1}{q}$, for example, take $f(x)=\frac{1}{8}$
\\You can take x=$\frac{1}{8},\frac{3}{8},\frac{5}{8},\frac{7}{8}$. Any other values would not be in the range. 
\\Given some (small) $\delta$, consider $x\in\mathbb{Q}$
\[\text{Does }|x-x_0|<\delta\implies |f(x)-0|<\epsilon\]
\\For example, let $\epsilon = \frac{1}{n_0}$
\\Pick $f(x)=\frac{1}{n_0+1}$ so that $|f(x)-0|<\epsilon$
\\Now, for all values of f(x) where $n_1>n_0$, it is clear $|\frac{1}{n_1}|<|\frac{1}{n_0}|<\epsilon$
\\Thus, we only need to worry about values where $n_2<n_0$. However, there are only finitely many values here, thus we can always set a value of $\delta$ small enough so all elements are equal to 0. $|x-x_0|<\delta$
\\Def: Let $f:A\subseteq \mathbb{R\to R}$ for a function, and $I_c(x_0)\subseteq A.$ We say that f is continuous at $x_0$ if 
\[\lim_{x\to x_0}f(x)=f(x_0)\]
\\\textbf{Remark}
\\Being continuous at $x_0$ means 3 things
\\\begin{enumerate}
    \item $\lim_{x\to x_0}f(x)$ exists
    \item f is defined at $x_0$
    \item $\lim_{x\to x_0}f(x)=f(x_0)$
\end{enumerate}
\noindent \textbf{Examples}
\\f(x)=c, continuous everywhere
$f:\mathbb{R\to R}, x\to x$, continuous everywhere
$f:\mathbb{R\to R}, x\to x^2$, continuous everywhere
\\Heaviside Function, everywhere but 0
\\Dirichlet function, nowhere
\\Thomae-Riemann function, only on the irrational numbers (rational numbers have a limit, but the limit does not equal 0)
\\\textbf{Definition: One-side limit}
\\Let $f:A\to \mathbb{R}$, and $(x_0-c,x_0)\subseteq A$
\\for some $c>0$, we say that 
\[\lim_{x\to x_0^-}f(x)=L\]
if 
\[\forall \epsilon>0, \exists \delta >0, st, 0<x_0-x<\delta \implies |f(x_0)-L|<\epsilon\]
\\\textbf{Theorem}
\[\lim_{x\to x_0}f(x)=L\iff \lim_{x\to x_0^-}f(x)=L=\lim_{x\to x_0^+}f(x)\]
\\Types of discontinuities
\\\begin{enumerate}
    \item Type I: f is not continuous at $x_0$, but both $\lim_{x\to x_0^-}f(x)\text{ and }\lim_{x\to x_0^+}f(x)$ exist.
    \item Type II: Either $\lim_{x\to x_0^-}f(x)\text{ or }\lim_{x\to x_0^+}f(x)$ does not exist
\end{enumerate}
\noindent \textbf{Example}
\[f:\mathbb{R\to R}\]
\[f(x)=\begin{cases}
    0,\ x=0\\
    \sin(\frac{1}{x}),x\not = 0
\end{cases}\]
\\$\frac{1}{x}$ can be very big at 0, and changes a lot with small differences in 0. Because this oscillates so much, there can be no limit at 0 from above or below. Type II discountinuity.
\end{document}
